\newpage
\section{Conclusion et perspectives}
\label{sec:conclusion}

\noindent Dans cette section, nous résumons la réalisation du projet et nous présentons également les extensions et améliorations possibles du projet.

\subsection{Résumé du travail réalisé}

\paragraph{}Ce projet nous a été très bénéfique tout d'abord dans l'aspect informatique : on a developpé nos connaissances en JAVA que ca soit en console avec les classes de données et de traitement avec la manipulation d'ArrayList avec un iterator ou en interface graphique ou on a manipulé differents elements d'affichage et de traitement.
\paragraph{}Le projet nous a aussi permis de mieux s'organiser en groupe car c'est plus compliqué de rassembler nos classes et de se répartir les taches a 3 alors qu'on est habituellement en binôme, on a donc dû utiliser des outils comme GitHub pour se partager nos fichiers ainsi que Discord pour s'appeler pendant le confinement et faire des partages d'écran.


\subsection{Améliorations possibles du projet}

\paragraph{}Pour améliorer le projet on pourrait ajouter les fonctionnalités avancées comme le swap, le future ou le forex.

\subsection{Sites et ressources utilisées}

\begin{itemize}
\item \url{https://stackoverflow.com/}
\item \url{https://www.java-forums.org/}
\item \url{https://www.vogella.com/tutorials/JFreeChart/article.html}
\item \url{https://docs.oracle.com/cd/E17802_01/j2se/javase/6/jcp/beta/apidiffs/java/sql/DataSet.html}
\item \url{https://www.boraji.com/jfreechart-xy-line-chart-example}
\item \url{https://openclassrooms.com/forum/}
\item \url{http://www.jfree.org/jfreechart/download.html}
\item \url{https://github.com/}
\end{itemize}
