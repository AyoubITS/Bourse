\newpage
\section{Introduction}
\label{sec:introduction}

Dans cette section, nous allons présenter brièvement l'objectif du projet 


\subsection{Contexte du projet}

% Un paragraphe naturel avec un saut de ligne

\paragraph{}Nous avons choisi le sujet de la
bourse car c'est une très bonne manière d'aborder le sujet qu'est la finance. En effet, tous trois
étudiants en l2 MIPI option informatique, avons la possibilité d'obtenir de nouvelles capacités de
programmation tout en ayant la possibilité de nous intéresser à un domaine divergent à nos études
qu'est la finance.
\paragraph{}C'est donc une opportunité d'allier une progression dans le domaine de la
programmation à l’apprentissage de nouvelles informations qui nous était jusqu'à l'heure plutôt
méconnues. L'argent, la finance, l 'économie faisant partis des principaux domaines qui régissent ce
monde, c'est naturellement que nous nous sommes tournés vers ce sujet.

\subsection{Objectif du projet}
\paragraph{ L'objectif du projet est de réaliser une interface de simulation boursère.}
\paragraph{} L'interface permet à l'utilisateur de choisir une des 20 entreprises créees par nous même et de surveiller la variation du prix d'action. Il peut également avoir des informations sur les evenements de la semaine ainsi qu'effectuer des achats et ventes à l'aide de son portefeuille.
\paragraph{ En bref le projet est comme un logiciel de trading pour utilisateur "débutant" (car sans fonctionnalités avancées).}
\subsection{Organisation du rapport}
\paragraph{} Tout au long du rapport nous allons présenter et expliquer nos différentes classes ainsi que notre interface graphique. 
